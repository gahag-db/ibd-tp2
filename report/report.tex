\documentclass{article}

\usepackage[margin=1in]{geometry}
\usepackage[portuges]{babel}
\usepackage[utf8]{inputenc}
\usepackage[hidelinks]{hyperref}
\usepackage{amsmath}
\usepackage{indentfirst}
\usepackage{minted}
\usemintedstyle{manni}
\usepackage{graphicx}
\usepackage{multicol,lipsum}
\usepackage{svg}
\usepackage[T1]{fontenc}
\usepackage{microtype}
\usepackage{algorithm}
\usepackage[noend]{algpseudocode}
\usepackage{mathtools}
\DeclarePairedDelimiter{\ceil}{\lceil}{\rceil}
\DisableLigatures[>,<]{encoding = T1,family=tt*} %
\makeatletter
\def\BState{\State\hskip-\ALG@thistlm}
\makeatother

\begin{document}
%\maketitle

\begin{titlepage}
	\begin{center}
	  
		\Huge{Universidade Federal de Minas Gerais}\\			
		\vspace{15pt}
    \vspace{95pt}
    \textbf{\LARGE{Trabalho Prático 2}}\\
		%\title{{\large{Título}}}
		\vspace{3,5cm}
    \begin{figure}[h]
      \begin{center}
        \includegraphics[scale = 0.50]{ufmg.png}
      \end{center}
     \label{fig:graph}
    \end{figure}
    	  
	\end{center}
  
  \begin{flushleft}
		\begin{tabbing}
      \textbf {Introdução a Banco de Dados}\\
      \\
			  Fernanda Guimarães de Araújo\\
	      Gabriel Silva Bastos\\
        Yasmim Cavalcanti de Rezende
	    \end{tabbing}
  \end{flushleft}
  
	  \vspace{1cm} 
	\begin{center}
		\vspace{\fill}
		Junho\\
		2018
	\end{center}
\end{titlepage}


\newpage
% % % % % % % % % % % % % % % % % % % % % % % % % %

\newpage
\pagenumbering{arabic}
% % % % % % % % % % % % % % % % % % % % % % % % % % %
\section{Introdução}
O objetivo deste trabalho foi implementar um banco de dados a partir da seguinte base de dados
governamental: compras públicas do governo federal. O projeto será apresentado em aula. Serão apresentados o diagrama
ER, o esquema relacional e as consultas (comparando as duas versões). O banco de dados foi instanciado
com dados reais.
\section{Descrição dos Dados}
Os dados possuem números exatos (inteiros), \textit{varchar} e datas.
\par Em realação à organização, pode-se dizer que, antes dos filtros, os dados estavam bem
desorganizados,com referência a dados inexistentes, ou seja, uma quebra na restrição de integridade referencial.
Além disso, havia muita informação desnecessária para o estudo do banco. Dessa forma, com os filtros,
pegamos apenas as informações relevantes para a análise proposta do banco de dados. 
\section{Características do Banco de Dados}
\subsection { Diagrama entidade-relacionamento}
\subsection {Esquema Relacional}
\noindent
O esquema relacional, na sua forma textual, é o seguinte: \\[10pt]
Fornecedores : (\underline{id}, cnpj, nome, uf) \\
Orgaos: (\underline{id}, nome, esfera) \\
Municipios: (\underline{id}, nome, nome, sigla,nome\_uf, sigla\_uf) \\
Modalidades: (\underline{id}, descricao) \\
UASGs: (\underline{id}, nome, CEP, municipio, orgao) \\
Licitacoes: (\underline{id}, modalidade, uasg) \\
Contratos: (\underline{id}, licitacao, fornecedor, data) \\[5pt]
Contratos.licitacao $\rightarrow$ Licitacao.id \\
Contratos.fornecedor $\rightarrow$ Fornecedor.id \\
Licitacoes.modalidade $\rightarrow$ Modalidade.id \\
UASGs.municipio $\rightarrow$ Municipio.id \\
UASGs.orgao $\rightarrow$ Orgao.id


\section{Comandos SQL}

\noindent
\subsection{Queries que envolvem agregações de duas ou mais juncoes}
\begin{itemize}
\item O número de licitações que envolvem o órgão "Senado Federal".
  \begin{minted}[mathescape, gobble=2]{sql}
    SELECT Count(*)
    FROM Licitacoes
      JOIN UASGs
      ON (Licitacoes.uasg = UASGs.id)
      JOIN Orgaos
      ON (UASGs.orgao = Orgaos.id)
      WHERE Orgaos.nome = 'SENADO FEDERAL'
  \end{minted}
  \pagebreak
    \begin{minted}[mathescape, gobble=2]{sql}
      SELECT Count(*)
      FROM Licitacoes, UASGs, Orgaos
      WHERE Licitacoes.uasg = UASGs.id
        AND UASGs.orgao = Orgaos.id
        AND Orgaos.nome = 'SENADO FEDERAL'


  \end{minted}

\item O ID e o número de contratos das licitações que envolvem dois ou mais contratos.
  \begin{minted}[mathescape, gobble=2]{sql}
    SELECT Licitacoes.id,COUNT(*)
    FROM Licitacoes, Contratos 
    WHERE Licitacoes.id = Contratos.licitacao
      GROUP BY Licitacoes.id
      HAVING COUNT(*) > 2
      ORDER BY COUNT(*) DESC
  \end{minted}
    
  \begin{minted}[mathescape, gobble=2]{sql}
    SELECT Licitacoes.id, COUNT(*)
    FROM Licitacoes
      INNER JOIN Contratos ON (Licitacoes.id = Contratos.licitacao)
    GROUP BY Licitacoes.id
      HAVING COUNT(*) > 2
    ORDER BY COUNT(*) DESC
  \end{minted}
  
\end{itemize}
 
\subsection{Queries com junção de duas relações}
\begin{itemize}
\item O ID e o nome dos fornecedores do DF com contratações de 2012.
  \begin{minted}[mathescape, gobble=2]{sql}
    SELECT Fornecedores.nome, Fornecedores.id
    FROM Fornecedores, Contratos 
    WHERE Fornecedores.id = Contratos.fornecedor
      AND YEAR(Contratos.data) = 2012
      AND Fornecedores.uf = 'DF'

    SELECT Fornecedores.nome, Fornecedores.id
    FROM Fornecedores
      INNER JOIN Contratos ON (Fornecedores.id = Contratos.fornecedor)
    WHERE YEAR(Contratos.data) = 2012
      AND Fornecedores.uf = 'DF'
  
\end{minted}  
\item O ID das contratações que nao possuem licitações cuja modalidade é "PREGÃO".
  \begin{minted}[mathescape, gobble=2]{sql}
    SELECT Contratos.id
    FROM Contratos
    WHERE NOT EXISTS (
    SELECT L.id
    FROM Licitacoes L
      INNER JOIN Modalidades M
      ON L.modalidade = M.id
    WHERE Contratos.licitacao = L.id
      AND M.descricao = 'PREGÃO'
    )
  \end{minted}
  \pagebreak
  \begin{minted}[mathescape, gobble=2]{sql}      
    SELECT Contratos.id
    FROM Contratos
    WHERE NOT EXISTS (
      SELECT L.id
      FROM Licitacoes L, 
        Modalidades M WHERE L.modalidade = M.id
      AND Contratos.licitacao = L.id
      AND M.descricao = 'PREGÃO'
)

  \end{minted}
\item O ID e o nome dos fornecedores que não possuem contratos iniciados em 2012?.
  \begin{minted}[mathescape, gobble=2]{sql}
    SELECT Fornecedores.nome, Fornecedores.id
    FROM Fornecedores
      INNER JOIN Contratos
      ON (Fornecedores.id = Contratos.fornecedor)
      WHERE YEAR(Contratos.data) >= 2012

    SELECT Fornecedores.nome, Fornecedores.id
    FROM Fornecedores,
      Contratos WHERE (Fornecedores.id = Contratos.fornecedor)
    AND YEAR(Contratos.data) >= 2012 
      
\end{minted}  
    
\end{itemize}

\subsection{Queries que envolvem junção de três ou mais relações}
\begin{itemize}
\item O ID das licitações envolvidas com órgãos que contém no nome "Turismo".
  \begin{minted}[mathescape, gobble=2]{sql}
    SELECT Licitacoes.id
    FROM Licitacoes, UASGs, Orgaos
    WHERE Licitacoes.uasg = UASGs.id
      AND UASGs.orgao = Orgaos.id
      AND Orgaos.nome LIKE '%Turismo%'

    SELECT Licitacoes.id
    FROM Licitacoes
      INNER JOIN UASGs ON (Licitacoes.uasg = UASGs.id)
      INNER JOIN Orgaos ON (UASGs.orgao = Orgaos.id) 
    WHERE Orgaos.nome LIKE '%Turismo%'
  
\end{minted}  
\item O nome das UASGs que possuem licitações no município "Belo Horizonte" cuja modalidade é "PREGÃO".
  \begin{minted}[mathescape, gobble=2]{sql}
    SELECT DISTINCT UASGs.nome
    FROM Licitacoes, UASGs, Municipios, Modalidades
    WHERE Licitacoes.uasg = UASGs.id
      AND UASGs.municipio = Municipios.id    
      AND Modalidades.id = Licitacoes.modalidade 
      AND Municipios.nome = 'Belo Horizonte'
      AND Modalidades.descricao = 'PREGÃO'
  \end{minted}
  \pagebreak
\begin{minted}[mathescape, gobble=2]{sql}
    SELECT DISTINCT UASGs.nome
    FROM Licitacoes INNER JOIN UASGs
      ON Licitacoes.uasg = UASGs.id
      INNER JOIN Municipios
        ON UASGs.municipio = Municipios.id
      INNER JOIN Modalidades
        ON Modalidades.id = Licitacoes.modalidade
      AND Municipios.nome = 'Belo Horizonte'
      AND Modalidades.descricao = 'PREGÃO'
      
  \end{minted}

\item O ID das licitações que envolvem o ministério do turismo e cujo município é "Brasília".
  \begin{minted}[mathescape, gobble=2]{sql}
    SELECT Licitacoes.id
    FROM Licitacoes, UASGs, Orgaos, Municipios
    WHERE Licitacoes.uasg = UASGs.id
      AND UASGs.orgao = Orgaos.id
      AND UASGs.municipio = Municipios.id 
      AND Orgaos.nome = 'MINISTERIO DO TURISMO'
      AND Municipios.nome = 'BRASÍLIA'

    SELECT Licitacoes.id
    FROM Licitacoes
      INNER JOIN UASGs ON (Licitacoes.uasg = UASGs.id)
      INNER JOIN Orgaos ON (UASGs.orgao = Orgaos.id)
      INNER JOIN Municipios ON (UASGs.municipio = Municipios.id)
    WHERE Orgaos.nome = 'MINISTERIO DO TURISMO'
      AND Municipios.nome = 'BRASÍLIA'
  \end{minted}
    
\end{itemize}

\subsection{Queries envolvendo projeção e seleção}
\begin{itemize}
\item O ID e o nome dos fornecedores que não possuem contratos iniciados antes de "2007-12-28".
  \begin{minted}[mathescape, gobble=2]{sql}
    SELECT Fornecedores.id, Fornecedores.nome
    FROM Fornecedores
    WHERE Fornecedores.id NOT IN (
      SELECT Fornecedores.id
      FROM Fornecedores
        INNER JOIN Contratos
          ON Fornecedores.id = Contratos.fornecedor
      WHERE Contratos.data < '2007-12-28')
      
    SELECT Fornecedores.id, Fornecedores.nome
    FROM Fornecedores
      LEFT JOIN (
        SELECT fornecedor, id
        FROM Contratos
        WHERE data < '2007-12-28'
      ) AS datab407
        ON Fornecedores.id = datab407.fornecedor 
    WHERE datab407.id IS NULL;
    \end{minted}
\item O ID e o nome dos forncedores não vem de MG.
  \begin{minted}[mathescape, gobble=2]{sql}
    SELECT Fornecedores.id, Fornecedores.nome
    FROM Fornecedores
    WHERE Fornecedores.uf <> 'MG'

    SELECT Fornecedores.id, Fornecedores.nome
    FROM Fornecedores
    WHERE Fornecedores.id NOT IN (
      SELECT Fornecedores.id
      FROM Fornecedores
	    WHERE Fornecedores.uf = 'MG'
    )

    \end{minted}
\end{itemize}

\section{Tempo de Execução}
O gráfico mostra a comparação dos tempos de execução das duas versões das queries.
O tempo foi o \textit{user time} do Linux.
\begin{figure}[H]
  \begin{center}  
    \includesvg[width=.90\linewidth]{graphs/graph0}    
  \end{center}  
  \label{fig:graph}
\end{figure}

\section{Autoavaliação dos Membros}
Os membros se autoavaliaram com nota máxima para cada um.
\begin{itemize}
\item \textbf{Fernanda:} Participou na aquisição e normalização do Banco de Dados, na produção do relatório em LaTex e na elaboração das queries.
\item \textbf{Gabriel:} Participou na aquisição e normalização do Banco de Dados, na produção do gráfico com o tempo de execução, e na elaboração das queries.
\item \textbf{Yasmim:} Participou na elaboração das queries e do diagrama ER, além da produção do relatório em LaTex
\end{itemize}
\section{Conclusão}

\bibliographystyle{plain}
\bibliography{refs}


\end{document}



