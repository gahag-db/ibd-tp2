\documentclass[a4paper, 12pt]{article}

\usepackage[portuges]{babel}
\usepackage[utf8]{inputenc}
\usepackage[hidelinks]{hyperref}
\usepackage{amsmath}
\usepackage{indentfirst}
\usepackage{graphicx}
\usepackage{multicol,lipsum}
\usepackage{svg}
\usepackage[T1]{fontenc}
\usepackage{microtype}
\usepackage{algorithm}
\usepackage[noend]{algpseudocode}
\usepackage{mathtools}
\DeclarePairedDelimiter{\ceil}{\lceil}{\rceil}
\DisableLigatures[>,<]{encoding = T1,family=tt*} %
\makeatletter
\def\BState{\State\hskip-\ALG@thistlm}
\makeatother

\begin{document}
%\maketitle

\begin{titlepage}
	\begin{center}
	  
		\Huge{Universidade Federal de Minas Gerais}\\			
		\vspace{15pt}
    \vspace{95pt}
    \textbf{\LARGE{Trabalho Prático 2}}\\
		%\title{{\large{Título}}}
		\vspace{3,5cm}
    \begin{figure}[h]
      \begin{center}
        \includegraphics[scale = 0.50]{ufmg.png}
      \end{center}
     \label{fig:graph}
    \end{figure}
    	  
	\end{center}
  
  \begin{flushleft}
		\begin{tabbing}
      \textbf {Introdução a Banco de Dados}\\
      \\
			  Fernanda Guimarães de Araújo\\
	      Gabriel Silva Bastos\\
        Yasmim Cavalcanti de Rezende
	    \end{tabbing}
  \end{flushleft}
  
	  \vspace{1cm} 
	\begin{center}
		\vspace{\fill}
		Junho\\
		2018
	\end{center}
\end{titlepage}


\newpage
% % % % % % % % % % % % % % % % % % % % % % % % % %

\newpage
\pagenumbering{arabic}
% % % % % % % % % % % % % % % % % % % % % % % % % % %
\section{Introdução}
O objetivo deste trabalho foi implementar um banco de dados a partir da seguinte base de dados
governamental: compras públicas do governo federal. O projeto será apresentado em aula. Serão apresentados o diagrama
ER, o esquema relacional e as consultas (comparando as duas versões). O banco de dados foi instanciado
com dados reais.


\section{Características do Banco de Dados}
\subsection { Diagrama entidade-relacionamento}
\subsection {Esquema Relacional}
\noindent
O esquema relacional, na sua forma textual, é o seguinte: \\[10pt]
Fornecedores : (\underline{id}, cnpj, nome, uf) \\
Orgaos: (\underline{id}, nome, esfera) \\
Municipios: (\underline{id}, nome, nome, sigla,nome\_uf, sigla\_uf) \\
Modalidades: (\underline{id}, descricao) \\
UASGs: (\underline{id}, nome, CEP, municipio, orgao) \\
Licitacoes: (\underline{id}, modalidade, uasg) \\
Contratos: (\underline{id}, licitacao, fornecedor, data) \\[5pt]
Contratos.licitacao $\rightarrow$ Licitacao.id \\
Contratos.fornecedor $\rightarrow$ Fornecedor.id \\
Licitacoes.modalidade $\rightarrow$ Modalidade.id \\
UASGs.municipio $\rightarrow$ Municipio.id \\
UASGs.orgao $\rightarrow$ Orgao.id


\section{Comandos SQL}

\noindent
 1 - O número de licitações que envolvem o órgão "Senado Federal".\\
 2 - O ID e o nome dos fornecedores do DF com contratações de 2012.\\
 3 - O ID das licitações envolvidas com órgãos que contém no nome "Turismo".\\
 4 - O ID das licitações cuja modalidade é "PREGÃO" ou "TOMADA DE PREÇOS".\\
 5 - O ID das licitações que envolvem o ministério do turismo e cujo município é "Brasília".\\
 6 - O ID e o número de contratos das licitações que envolvem dois ou mais contratos.\\
 7 - O ID das contratações que nao possuem licitações cuja modalidade é "PREGÃO".\\
 8 - O ID e o nome dos fornecedores que não possuem contratos iniciados em 2012? [erro].\\
 9 - O ID e o nome dos fornecedores que não possuem contratos iniciados antes de "2007-12-28".\\
10 - O ID e o nome dos forncedores não vem de MG.

\section{Conclusão}
O algoritmo e as estruturas de dados escolhidos, a intercalação balanceada monofásica e os vetores
(mostrados na análise espacial), respectivamente, se mostraram bastante eficientes,
no sentido em que a análise experimental ficou quase \textit{n $\cdot$ log n}. Assim,
os arquivos são ordenados em memória externa, respeitando o limite dado.

\bibliographystyle{plain}
\bibliography{refs}


\end{document}



